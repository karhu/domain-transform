\documentclass[11pt,a4paper]{article}
\title{How To Write Fast Numerical Code\\ Course Project}
\author{Julia Pe\v{c}erska, student number 12-947-396 \\ Adrian Blumer, student number xx-xxx-xxx \\ Pascal Sp\"{o}rri, student number xx-xxx-xxx}
\date{10.04.2013}
\usepackage[compact,explicit]{titlesec}
\usepackage{graphicx}
\usepackage[normalem]{ulem}
\usepackage{multirow}
\usepackage{listings}
\usepackage{tabularx}
\usepackage{float}
\usepackage{enumerate}
\usepackage{multirow}

\newcommand{\todo}[1]{\vskip 5mm\noindent{\bf TODO}: {\it #1}}

\begin{document}
%\titleformat{\section}{\large\bfseries}%{}{0pt}%{\thesection}%{Exercise \thesection \}
\maketitle
\section{Cost analysis}

In the Table~\ref{tab:cost_an} below are included all of the operations used in the source code for filtering methods. The counting was done in the source code, no instrumentation used.

The following variables are used in Table~\ref{tab:cost_an}: $W$ - width of the processed image, $H$ - height of the processed image, $N$ - number of filtering iterations.


\begin{table}[h]
\label{tab:cost_an}
\centering
\begin{tabular}{| l | l | l |} \hline 
Operation & Recursive filtering & Normalized convolution \\ \hline
add & $(6 + 12 N) W H$ & $(9 + 2 N) W H$ \\ \hline
subtract & $(6 + 12 N) W H$ & $(6 + 4 N) W H$ \\ \hline
multiply & $(2 + 12 N) W H + 2 N$ & $W H + 3 N$ \\ \hline
divide & $3 N + 1$ & $6 N + 1$ \\ \hline
fabs & $6 W H$ & $6 W H$ \\ \hline
sqrt & $4 N$ & $3 N$ \\ \hline
powf & $2 N W H + 2 N$ & $2 N$ \\ \hline
exp & $2 N$ & $0$ \\ \hline
\end{tabular}
\caption{Floating point operations performed in the code}
\end{table}

\end{document}